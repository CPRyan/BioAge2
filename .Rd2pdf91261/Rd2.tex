\documentclass[a4paper]{book}
\usepackage[times,inconsolata,hyper]{Rd}
\usepackage{makeidx}
\usepackage[utf8]{inputenc} % @SET ENCODING@
% \usepackage{graphicx} % @USE GRAPHICX@
\makeindex{}
\begin{document}
\chapter*{}
\begin{center}
{\textbf{\huge Package `BioAge'}}
\par\bigskip{\large \today}
\end{center}
\begin{description}
\raggedright{}
\inputencoding{utf8}
\item[Type]\AsIs{Package}
\item[Title]\AsIs{Biological Age Calculations Using Several Biomarker Algorithms}
\item[Version]\AsIs{0.1.0}
\item[Author]\AsIs{Dayoon Kwon}
\item[Maintainer]\AsIs{Dayoon Kwon }\email{dk1943@nyu.edu}\AsIs{}
\item[Description]\AsIs{This package measures biological aging using data from the National Health and Nutrition Examination Survey (NHANES). The package uses published biomarker algorithms to calculate three biological age measures: Klemera-Doubal Method biological age, phenotypic age, and homeostatic dysregulation.}
\item[License]\AsIs{GPL-3}
\item[Encoding]\AsIs{UTF-8}
\item[LazyData]\AsIs{true}
\item[RoxygenNote]\AsIs{7.1.0}
\item[Depends]\AsIs{R (>= 2.10)}
\item[Imports]\AsIs{dplyr,
flexsurv}
\item[Suggests]\AsIs{testthat,
knitr,
rmarkdown}
\item[VignetteBuilder]\AsIs{knitr}
\end{description}
\Rdcontents{\R{} topics documented:}
\inputencoding{utf8}
\HeaderA{bioage\_calc}{bioage\_calc}{bioage.Rul.calc}
%
\begin{Description}\relax
Calculate Klemera-Doubal Method (KDM) biological age
\end{Description}
%
\begin{Usage}
\begin{verbatim}
bioage_calc(data, age, biomarkers, fit = NULL, s_ba2 = NULL)
\end{verbatim}
\end{Usage}
%
\begin{Arguments}
\begin{ldescription}
\item[\code{data}] The dataset for calculating KDM bioage

\item[\code{age}] A character vector (length=1) indicating the name of the variable for chronological age

\item[\code{biomarkers}] A character vector indicating the names of the variables for the biomarkers to use in calculating KDM bioage

\item[\code{fit}] An S3 object for model fit. If the value is NULL, then the parameters to use for training KDM bioage are calculated

\item[\code{s\_ba2}] A particular fit parameter. Advanced users can modify this parameter to control the variance of bioage
\end{ldescription}
\end{Arguments}
%
\begin{Details}\relax
Calculate KDM biological age
\end{Details}
%
\begin{Value}
An object of class "bioage". This object is a list with two elements (data and fit)
\end{Value}
%
\begin{Examples}
\begin{ExampleCode}
#Train KDM bioage parameters
train = bioage_calc(nhanes3,age="age",
                    biomarkers=c("fev","sbp","totchol","hba1c","albumin","creat","lncrp","alp","bun"))

#Use training data to calculate KDM bioage
bioage = bioage_calc(nhanes,age="age",
                     biomarkers=c("fev","sbp","totchol","hba1c","albumin","creat","lncrp","alp","bun"),
                     fit=train$fit,
                     s_ba2=train$fit$s_ba2)

#Extract bioage dataset
data = bioage$data


\end{ExampleCode}
\end{Examples}
\inputencoding{utf8}
\HeaderA{bioage\_nhanes}{bioage\_nhanes}{bioage.Rul.nhanes}
%
\begin{Description}\relax
Calculate Klemera-Doubal Method (KDM) biological age using NHANES dataset
\end{Description}
%
\begin{Usage}
\begin{verbatim}
bioage_nhanes(biomarkers)
\end{verbatim}
\end{Usage}
%
\begin{Arguments}
\begin{ldescription}
\item[\code{biomarkers}] A character vector indicating the names of the variables for the biomarkers to use in calculating bioage
\end{ldescription}
\end{Arguments}
%
\begin{Details}\relax
Calculate KDM biological age using NHANES dataset
\end{Details}
%
\begin{Value}
An object of class "bioage". This object is a list with two elements (data and fit)
\end{Value}
%
\begin{Examples}
\begin{ExampleCode}
#Calculate KDM bioage
bioage = bioage_nhanes(biomarkers=c("fev","sbp","totchol","hba1c","albumin","creat","lncrp","alp","bun"))

#Extract bioage dataset
data = bioage$data


\end{ExampleCode}
\end{Examples}
\inputencoding{utf8}
\HeaderA{hd\_calc}{hd\_calc}{hd.Rul.calc}
%
\begin{Description}\relax
Calculate homeostatic dysregulation (HD)
\end{Description}
%
\begin{Usage}
\begin{verbatim}
hd_calc(data, reference, biomarkers)
\end{verbatim}
\end{Usage}
%
\begin{Arguments}
\begin{ldescription}
\item[\code{data}] The dataset for calculating HD

\item[\code{reference}] The reference dataset for calculating HD

\item[\code{biomarkers}] A character vector indicating the names of the variables for the biomarkers to use in calculating HD
\end{ldescription}
\end{Arguments}
%
\begin{Details}\relax
Calculate HD
\end{Details}
%
\begin{Value}
An object of class "hd". This object is a list with two elements (data and fit)
\end{Value}
%
\begin{Examples}
\begin{ExampleCode}
#Calculate HD
hd = hd_calc(nhanes,nhanes3,
             biomarkers=c("albumin_gL","lymph","mcv","glucose_mmol","rdw","creat_umol","lncrp","alp","wbc"))

#Extract bioage dataset
data = hd$data


\end{ExampleCode}
\end{Examples}
\inputencoding{utf8}
\HeaderA{hd\_nhanes}{hd\_nhanes}{hd.Rul.nhanes}
%
\begin{Description}\relax
Calculate homeostatic dysregulation (HD) using NHANES dataset
\end{Description}
%
\begin{Usage}
\begin{verbatim}
hd_nhanes(biomarkers)
\end{verbatim}
\end{Usage}
%
\begin{Arguments}
\begin{ldescription}
\item[\code{biomarkers}] A character vector indicating the names of the variables for the biomarkers to use in calculating HD
\end{ldescription}
\end{Arguments}
%
\begin{Details}\relax
Calculate HD using NHANES dataset
\end{Details}
%
\begin{Value}
An object of class "hd". This object is a list with two elements (data and fit)
\end{Value}
%
\begin{Examples}
\begin{ExampleCode}
#Calculate HD
hd = hd_nhanes(biomarkers=c("albumin","lymph","mcv","glucose","rdw","creat","lncrp","alp","wbc"))

#Extract HD dataset
data = hd$data


\end{ExampleCode}
\end{Examples}
\inputencoding{utf8}
\HeaderA{phenoage\_calc}{phenoage\_calc}{phenoage.Rul.calc}
%
\begin{Description}\relax
Calculate Levine's phenotypic age
\end{Description}
%
\begin{Usage}
\begin{verbatim}
phenoage_calc(data, age, time, status, biomarkers, fit = NULL)
\end{verbatim}
\end{Usage}
%
\begin{Arguments}
\begin{ldescription}
\item[\code{data}] The dataset for calculating phenoage

\item[\code{age}] A character vector (length=1) indicating the name of the variable for chronological age

\item[\code{time}] A character vector (length=1) indicating the name of the variable for survival time

\item[\code{status}] A character vector (length=1) indicating the name of the variable for survival status

\item[\code{biomarkers}] A character vector indicating the names of the variables for the biomarkers to use in calculating phenoage

\item[\code{fit}] An S3 object for model fit. If the value is NULL, then the parameters to use for training phenoage are calculated
\end{ldescription}
\end{Arguments}
%
\begin{Details}\relax
Calculate phenotypic age
\end{Details}
%
\begin{Value}
An object of class "phenoage". This object is a list with two elements (data and fit)
\end{Value}
%
\begin{Examples}
\begin{ExampleCode}
#Train phenoage parameters
train = phenoage_calc(nhanes3,age="age",
                      biomarkers=c("albumin_gL","lymph","mcv","glucose_mmol","rdw","creat_umol","lncrp","alp","wbc"))

#Use training data to calculate phenoage
phenoage = phenoage_calc(nhanes,age="age",
                         biomarkers=c("albumin_gL","lymph","mcv","glucose_mmol","rdw","creat_umol","lncrp","alp","wbc"),
                         fit=train$fit)

#Extract phenoage dataset
data = phenoage$data


\end{ExampleCode}
\end{Examples}
\inputencoding{utf8}
\HeaderA{phenoage\_nhanes}{phenoage\_nhanes}{phenoage.Rul.nhanes}
%
\begin{Description}\relax
Calculate Levine's phenotypic age using NHANES dataset
\end{Description}
%
\begin{Usage}
\begin{verbatim}
phenoage_nhanes(biomarkers)
\end{verbatim}
\end{Usage}
%
\begin{Arguments}
\begin{ldescription}
\item[\code{biomarkers}] A character vector indicating the names of the variables for the biomarkers to use in calculating phenoage
\end{ldescription}
\end{Arguments}
%
\begin{Details}\relax
Calculate Levine's phenotypic age using NHANES dataset
\end{Details}
%
\begin{Value}
An object of class "phenoage". This object is a list with two elements (data and fit)
\end{Value}
%
\begin{Examples}
\begin{ExampleCode}
#Calculate phenoage
phenoage = phenoage_nhanes(biomarkers=c("albumin_gL","lymph","mcv","glucose_mmol","rdw","creat_umol","lncrp","alp","wbc"))

#Extract phenoage dataset
data = phenoage$data


\end{ExampleCode}
\end{Examples}
\inputencoding{utf8}
\HeaderA{plot\_ba}{plot\_ba}{plot.Rul.ba}
%
\begin{Description}\relax
Plot correlations between biological aging measures and chronological age
\end{Description}
%
\begin{Usage}
\begin{verbatim}
plot_ba(data, agevar)
\end{verbatim}
\end{Usage}
%
\begin{Arguments}
\begin{ldescription}
\item[\code{data}] The dataset for plotting correlations

\item[\code{agevar}] A character vector indicating the names of the biological aging measures
\end{ldescription}
\end{Arguments}
%
\begin{Details}\relax
Plot correlations between biological aging measures and chronological age
\end{Details}
%
\begin{Note}\relax
Chronological age and gender variables need to be named "age" and "gender"
\end{Note}
%
\begin{Examples}
\begin{ExampleCode}
#Calculate phenoage
f1 = plot_ba(data = data, agevar = c("bioage", "phenoage", "hd"))

f1

\end{ExampleCode}
\end{Examples}
\inputencoding{utf8}
\HeaderA{plot\_baa}{plot\_baa}{plot.Rul.baa}
%
\begin{Description}\relax
Plot correlations between biological age advancement (BAA) and chronological age
\end{Description}
%
\begin{Usage}
\begin{verbatim}
plot_baa(data, agevar, labels, axis_type)
\end{verbatim}
\end{Usage}
%
\begin{Arguments}
\begin{ldescription}
\item[\code{data}] The dataset for plotting corplot

\item[\code{agevar}] A character vector indicating the names of the interested biological age measures

\item[\code{labels}] A character vector indicating the labels of the biological age measures
Values should be formatted for displaying along diagonal of the plot
Names should be used to match variables and order is preserved

\item[\code{axis\_type}] A character vector indicating the axis type (int or float)
Use variable name to define the axis type
\end{ldescription}
\end{Arguments}
%
\begin{Details}\relax
Plot correlations between BAA and chronological age
\end{Details}
%
\begin{Examples}
\begin{ExampleCode}
#Create corplot of BAA with chronologicl age
labels = c("bioage_advance0"="KDM\nBiological\nAge",
           "phenoage_advance0"="Levine\nPhenotypic\nAge",
           "bioage_advance"="Modified-KDM\nBiological\nAge",
           "phenoage_advance"="Modified-Levine\nPhenotypic\nAge",
           "hd" = "Mahalanobis\nDistance",
           "hd_log" = "Log\nMahalanobis\nDistance")

axis_type = c("bioage_advance0"="float",
              "phenoage_advance0"="float",
              "bioage_advance"="float",
              "phenoage_advance"="flot",
              "hd"="flot",
              "hd_log"="float")

f2 = plot_baa(data,
              agevar = c("bioage_advance0",
                            "phenoage_advance0",
                            "bioage_advance",
                            "phenoage_advance",
                            "hd",
                            "hd_log"),

              labels = c("bioage_advance0"="KDM\nBiological\nAge",
                         "phenoage_advance0"="Levine\nPhenotypic\nAge",
                         "bioage_advance"="Modified-KDM\nBiological\nAge",
                         "phenoage_advance"="Modified-Levine\nPhenotypic\nAge",
                         "hd" = "Mahalanobis\nDistance",
                         "hd_log" = "Log\nMahalanobis\nDistance"),

              axis_type = c("bioage_advance0"="float",
                            "phenoage_advance0"="float",
                            "bioage_advance"="float",
                            "phenoage_advance"="flot",
                            "hd"="flot",
                            "hd_log"="float"))

f2

\end{ExampleCode}
\end{Examples}
\inputencoding{utf8}
\HeaderA{surv\_table}{surv\_table}{surv.Rul.table}
%
\begin{Description}\relax
Mortality validation for biological age measures
\end{Description}
%
\begin{Usage}
\begin{verbatim}
surv_table(data, agevar, time, status)
\end{verbatim}
\end{Usage}
%
\begin{Arguments}
\begin{ldescription}
\item[\code{data}] The dataset for plotting corplot

\item[\code{agevar}] A character vector indicating the names of the interested biological age measures

\item[\code{time}] A character vector (length=1) indicating the name of the variable for survival time

\item[\code{status}] A character vector (length=1) indicating the name of the variable for survival status
\end{ldescription}
\end{Arguments}
%
\begin{Details}\relax
Mortality validation for biological age measures, adjusting for chronological age and gender
\end{Details}
%
\begin{Note}\relax
Chronological age, gender, and race/ethnicity variables need to be named "age", "gender", and "race"
\end{Note}
%
\begin{Examples}
\begin{ExampleCode}
table1 = surv_table(nhanes,
                    agevar = c("bioage_advance0","phenoage_advance0","bioage_advance","phenoage_advance","hd","hd_log"),
                    time = "permth_exm",
                    status = "mortstat")

table1

\end{ExampleCode}
\end{Examples}
\printindex{}
\end{document}
